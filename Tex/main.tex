\documentclass[a4paper, german, oneside]{scrbook}

\usepackage[german]{babel} %ngerman

\usepackage[utf8]{inputenc} 

\usepackage[T1]{fontenc}  
\usepackage{lmodern}

\usepackage{graphicx}

\usepackage{datetime}
\newdate{date}{15}{07}{2014}
\date{\displaydate{date}}

% Fußnote für Tabellen
\usepackage{tablefootnote}
\usepackage{amsmath}

% Auflistung in Text mit \begin{inparaenum}[(i)] und dann \item
\usepackage{paralist}
% Bessere Formatierung für Tabellen
\usepackage{booktabs}


\usepackage[babel,german=quotes]{csquotes}  % Deutsche Anführungszeichen mit \enquote{}

\usepackage[nottoc]{tocbibind}
\usepackage[style=authoryear, backend=biber, firstinits=false]{biblatex} % firstinits kürzt die Vornamen ab
\addbibresource{lit_ba_arbeit.bib}

\DefineBibliographyStrings{english}{andothers={et\ al\adddot}} % "u.a." zu "et al." 
% \DefineBibliographyStrings{english}{bibliography = {References}}
% \DefineBibliographyStrings{german}{bibliography = {Literaturverzeichnis}}


%richtige Reihenfolge bei mehreren Autoren
\DeclareNameAlias{sortname}{last-first}

%keine Klammern in Biblio
\renewbibmacro*{date+extrayear}{%
  \iffieldundef{year}
    {}
    {\printtext{\printdateextra}}}

\renewcommand*{\mkbibnamefirst}[1]{#1\addcomma} % #1 

% Commands für Textcite und Parencite mit Dropdownmenü
\newcommand{\citet}[1]{\textcite{#1}} 
\newcommand{\citep}[1]{\parencite{#1}}


% % Blockzitat mit Anführungszeichen
% \renewcommand*{\mkblockquote}[4]{\enquote{#1}#2#4#3}

\begin{document}
	\begin{titlepage}
	
		\begin{center}


		% Oberer Teil der Titelseite:
		% \includegraphics[width=0.15\textwidth]{./logo}\\[1cm]    

		\textsc{\LARGE Universität für Musik und Darstellende Kunst Graz}\\[1.5cm]

		% Titel der LV
		% \textsc{\Large Multivariate Datenanalyse (SS 2014)}\\[0.5cm]


		% Title
		\newcommand{\HRule}{\rule{\linewidth}{0.5mm}}
		\HRule \\[0.4cm]
		{ \huge \bfseries Irgendwas mit Publikum}\\[0.4cm]

		\HRule \\[2.5cm] % Bei Untertitel auf 0.5 zurücksetzen


		% \Large Untertitel	\\[2cm]
		% Evtl Wortumfang der Arbeit hier einfügen

		\Large \emph{Autor}:\\
		Thomas \textsc{Klebel}\\[0.1cm]
		\large 1073073\\[1cm]

		\Large \emph{Betreuerin:}\\
		Christa \textsc{Brüstle}\\[1cm]

		% \includegraphics[scale=0.5]{uni-graz_color.jpg}\\[1cm]    


		\vfill

		% Unterer Teil der Seite
		\Large{\today}\\[1cm]
		\normalsize{Zeichensatz mit \LaTeX}
		

		\end{center}

	\end{titlepage}

\tableofcontents

\chapter{Einleitung}


Im ersten Teil der Arbeit soll die Entwicklung des klassischen Konzerts, insbesondere im 19. Jahrhundert nachvollzogen werden. 

Im zweiten Teil der Arbeit soll der/die LeserIn einerseits in die Theoriewelt Bourdieus eingeführt werden. Andererseits werde ich verschiedene Modelle zur Sozialstrukturananlyse darstellen, sowie aktuelle empirische Untersuchungen zum Konzertpublikum rezipieren. Die empirischen Befunde, sowie die in Teil 1 dargelegte Entwicklung des klassischen Konzerts, sollen dann mittels der vorgestellten Theorien untersucht werden. 

Die Analyse wird sich an folgenden Fragestellungen orientieren:

\begin{itemize}
	\item Welche Rolle spielte das Konzert im Leben der Bevölkerung des 19. Jahrhunderts? %viel zu allgemein
	\item Wie stellt sich das Verhältnis zwischen Konzertbesuch und sozialer Lage im 19. Jahrhundert dar?
	\item Gibt es verbindende Merkmale, die den KonzertbesucherInnen des 19. Jahrhunderts gemein sind?
\end{itemize}

Die Erkenntnisse aus der Untersuchung des Konzertpublikums des 19. Jahrhunderts sollen den empirischen Befunden über das heutige Konzertpublikum gegenüber gestellt werden:

\begin{itemize}
	\item Aus welchen Personen(gruppen) setzt sich das heutige Konzertpublikum zusammen?
	\item Welche Gemeinsamkeiten und Unterschiede lassen sich im Vergleich mit dem Konzertpublikum des 19. Jahrhunderts finden?
\end{itemize}

\chapter{Historischer Teil}

\section{Das Musikleben im 18. Jahrhundert}

\section{Das Musikleben im 19. Jahrhundert}

\subsection{Historischer Überblick: Das 19. Jahrhundert}

\subsection{Oper und Konzert - Gemeinsamkeiten und Unterschiede}

\subsection{Konzerttypen im 19. Jahrhundert}

\subsection{Kanonbildung in der klassischen Musik}

\chapter{Soziologischer Teil}
\section{Bourdieu}

\section{Sozialstruktur}

\section{Empirische Erhebungen zum klassischen Konzertpublikum}


\chapter{Resümee}


blabla \cite{bourdieu_feinen_2012}

\printbibliography
\end{document}

