\documentclass[a4paper, german, oneside]{scrbook}

\usepackage[german]{babel} %ngerman

\usepackage[utf8]{inputenc} 

\usepackage[T1]{fontenc}  
\usepackage{lmodern}

\usepackage{graphicx}

\usepackage{datetime}
\newdate{date}{15}{07}{2014}
\date{\displaydate{date}}

% Fußnote für Tabellen
\usepackage{tablefootnote}
\usepackage{amsmath}

% Auflistung in Text mit \begin{inparaenum}[(i)] und dann \item
\usepackage{paralist}
% Bessere Formatierung für Tabellen
\usepackage{booktabs}


\usepackage[babel,german=quotes]{csquotes}  % Deutsche Anführungszeichen mit \enquote{}

\usepackage[nottoc]{tocbibind}
\usepackage[style=authoryear, backend=biber, firstinits=false]{biblatex} % firstinits kürzt die Vornamen ab
\addbibresource{lit_ba_arbeit.bib}

\DefineBibliographyStrings{english}{andothers={et\ al\adddot}} % "u.a." zu "et al." 
% \DefineBibliographyStrings{english}{bibliography = {References}}
% \DefineBibliographyStrings{german}{bibliography = {Literaturverzeichnis}}


%richtige Reihenfolge bei mehreren Autoren
\DeclareNameAlias{sortname}{last-first}

%keine Klammern in Biblio
\renewbibmacro*{date+extrayear}{%
  \iffieldundef{year}
    {}
    {\printtext{\printdateextra}}}

\renewcommand*{\mkbibnamefirst}[1]{#1\addcomma} % #1 

% Commands für Textcite und Parencite mit Dropdownmenü
\newcommand{\citet}[1]{\textcite{#1}} 
\newcommand{\citep}[1]{\parencite{#1}}

% fix weird problem with unicode
\DeclareUnicodeCharacter{00A0}{ }




% % Blockzitat mit Anführungszeichen
% \renewcommand*{\mkblockquote}[4]{\enquote{#1}#2#4#3}

\begin{document}
	\begin{titlepage}
	
		\begin{center}


		% Oberer Teil der Titelseite:
		% \includegraphics[width=0.15\textwidth]{./logo}\\[1cm]    

		\textsc{\LARGE Universität für Musik und Darstellende Kunst Graz}\\[1.5cm]

		% Titel der LV
		% \textsc{\Large Multivariate Datenanalyse (SS 2014)}\\[0.5cm]


		% Title
		\newcommand{\HRule}{\rule{\linewidth}{0.5mm}}
		\HRule \\[0.4cm]
		{ \huge \bfseries Irgendwas mit Publikum}\\[0.4cm]

		\HRule \\[2.5cm] % Bei Untertitel auf 0.5 zurücksetzen


		% \Large Untertitel	\\[2cm]
		% Evtl Wortumfang der Arbeit hier einfügen

		\Large \emph{Autor}:\\
		Thomas \textsc{Klebel}\\[0.1cm]
		\large 1073073\\[1cm]

		\Large \emph{Betreuerin:}\\
		Christa \textsc{Brüstle}\\[1cm]

		% \includegraphics[scale=0.5]{uni-graz_color.jpg}\\[1cm]    


		\vfill

		% Unterer Teil der Seite
		\Large{\today}\\[1cm]
		\normalsize{Zeichensatz mit \LaTeX}
		

		\end{center}

	\end{titlepage}

	
%%%%%%%%%%%%%%%%%%%%%%%%
% zähle die Titelseite nicht für die Seitenzahl
\clearpage
\setcounter{page}{1}
%%%%%%%%%%%%%%%%%%%%%%

\tableofcontents

\chapter*{Vorwort}
\addcontentsline{toc}{chapter}{Vorwort}

Im ersten Teil der Arbeit soll die Entwicklung des klassischen Konzerts, insbesondere im 19. Jahrhundert nachvollzogen werden. 

Im zweiten Teil der Arbeit soll der/die LeserIn einerseits in die Theoriewelt Bourdieus eingeführt werden. Andererseits werde ich verschiedene Modelle zur Sozialstrukturananlyse darstellen, sowie aktuelle empirische Untersuchungen zum Konzertpublikum rezipieren. Die empirischen Befunde, sowie die in Teil 1 dargelegte Entwicklung des klassischen Konzerts, sollen dann mittels der vorgestellten Theorien untersucht werden. 

Die Analyse wird sich an folgenden Fragestellungen orientieren:

\begin{itemize}
	\item Welche Rolle spielte das Konzert im Leben der Bevölkerung des 19. Jahrhunderts? %viel zu allgemein
	\item Wie stellt sich das Verhältnis zwischen Konzertbesuch und sozialer Lage im 19. Jahrhundert dar?
	\item Gibt es verbindende Merkmale, die den KonzertbesucherInnen des 19. Jahrhunderts gemein sind?
\end{itemize}

Die Erkenntnisse aus der Untersuchung des Konzertpublikums des 19. Jahrhunderts sollen den empirischen Befunden über das heutige Konzertpublikum gegenüber gestellt werden:

\begin{itemize}
	\item Aus welchen Personen(gruppen) setzt sich das heutige Konzertpublikum zusammen?
	\item Welche Gemeinsamkeiten und Unterschiede lassen sich im Vergleich mit dem Konzertpublikum des 19. Jahrhunderts finden?
\end{itemize}

%%%%%%%%%%%%%%%%%%%%%%%%%%%%%%%%%%%%%%%%%%%%%%%%%%%%%%%%%%%%%%%%%%%%%%%%%%%%%%%%%%%%%%%%%%%%%%%%%%%%%%%%%%%%%%%%%%%%%%%%%%%%%%%%%%%%%%%%%%%%%%%%%%%%%%%%%%
%%%%%%%%%%%%%%%%%%%%%%%%%%%%%%%%%%%%%%%%%%%%%%%%%%%%%%%%%%%%%%%%% Historischer Teil %%%%%%%%%%%%%%%%%%%%%%%%%%%%%%%%%%%%%%%%%%%%%%%%%%%%%%%%%%%%%%%%%%%%%%
%%%%%%%%%%%%%%%%%%%%%%%%%%%%%%%%%%%%%%%%%%%%%%%%%%%%%%%%%%%%%%%%%%%%%%%%%%%%%%%%%%%%%%%%%%%%%%%%%%%%%%%%%%%%%%%%%%%%%%%%%%%%%%%%%%%%%%%%%%%%%%%%%%%%%%%%%%


\part{Historischer Teil}
% begingropu und endgroup umfassen einen Bereich ohne Pagebreak

\begingroup
\renewcommand{\cleardoublepage}{}
\renewcommand{\clearpage}{}

\chapter{Einleitung}
Um die Zusammensetzung des heutigen KOnzertpublikums zu untersuchen erscheint es sinnvoll, zuerst einen Blick auf die Genese des Konzerts zu werfen: Wann ist das Konzert, so wie wir es heute kennen, entstanden? Aus welchen Formen ist es entstanden, welche Vorläufer gab es? Auf Basis der Antworten auf diese Fragen lassen sich dann die heutigen Entwicklungen diskutieren.

Um einen guten Überblick über die Genese des Konzerts zu bieten, werden im ersten Abschnitt die Anfänge der Entwicklung zum Konzert betrachtet werden. Im zweiten Abschnitt wird es um das Musikleben im 19. JAhrhundert gehen. Neben einem kurzen historischen Überblick über die wichtigsten Geschehnisse dieses Jahrhunderts sollen, neben den Unterschieden zwischen dem Opernpublikum und dem Konzertpublikum, hautpsächlich die für die Entstehung des \enquote{bürgerlichen} Konzerts bedeutenden Formen dargestellt werden: 
\begin{inparaenum}[(1)]
	\item das Liebhaberkonzert, 
	\item die geschlossene Konzertvereinigung, sowie 
	\item die professionelle Konzertgesellschaft.
\end{inparaenum}



\chapter{Das Musikleben im 18. Jahrhundert}
\label{18jh}
Mix aus verschiedenen Stücke, arien etc (Weber: 3)

Aus heutiger PErspektive erstaunlich erscheint die Tatsache, dass das Liebhaberkonzert im Grunde ein Potpourri aus den verschiedensten WErken war. Typischerweise eröffnete es mit einem Satz aus einer Sinfonie, auf den Lieder, Arien, und virtuose Instrumentalstücke folgten. (Literatur!!!)


\endgroup

\chapter{Das Musikleben im 19. Jahrhundert}
\label{19jh}

\section{Historischer Überblick: Das 19. Jahrhundert}
\label{histUberblick}
\section{Oper und Konzert - Gemeinsamkeiten und Unterschiede}
\label{operUndKonzert}
\section{Konzerttypen im 19. Jahrhundert}
\label{konzerttypen}
Bei der Analyse des Publikums lassen sich in der Literatur zwei Tendenzen erkennen. In den Werken von \citet{weber_music_2004} und \citet{bourdieu_feinen_2012} findet sich die klare Unterscheidung zwischen \emph{high culture} und \emph{popular culture} bei Weber, beziehungsweise \emph{legitimen}, \emph{mittlerem} und \emph{populärem Geschmack} bei Bourdieu. Bei neueren Untersuchungen, unter anderem bei \citet{gebesmair_grundzuge_2001} sowie \citet{muller_publikum_2014} wird diese Sicht teilweise kritisiert und relativiert. Ich möchte trotzdem mit einer Darstellung der Überlegungen bei Weber beginnen. Im späteren Verlauf der Arbeit sollen dann die Unzulänglichkeiten dieser Sichtweise diskutiert werden.

Liebhaberkonzert == benefit concert

\subsection{Das Liebhaberkonzert}
\label{liebhaber}
Obwohl der Terminus der \enquote{populären Musik} umstritten sein mag, so war doch das Liebhaberkonzert ein wichtiger Schritt auf dem Weg zur Form des Konzerts, so wie wir sie heute erleben. In der ersten HÄlfte des 19. Jahrhunderts gehörte es zu einer ----------, die maßgeblich an der Erweiterung des Publikums beteiligt war. Zu dieser ----- gehörten unter anderem eine institutionalisierte Form von Salons und Konzerten, die zusammen mit dem aufkeimendem Unternehmertum und der professionalisierung der MusikerInnen ein wachsendes Publikum aus der Aristokratie und der oberen Mittelschicht für sich gewinnen konnte.

Um die Rahmenbedinungen des Liebhaberkonzerts besser zu verstehen lohnt sich ein Blick in die Welt des Publikums dieser Konzerte.

Musik hatte für die aristokratischen Zirkel und die auf eine Verbesserung ihrer gesellschaftlichen Position bedachte obere Mittelschicht mehrere Funktionen. Ein erster Grund für die Tatsache, dass Musik eine immere wichtigere Rolle im Leben der Menschne einnahm war ihre Verankerung in der Familie. Zuerst einmal war eine musikalische ERziehung der Kinder ein guter Weg, sie Disziplin zu lehren: das ERlernen eines Musikinstruments fordert regelmäßige und konzentrierte Übung, folglich ein gewisses Maß an Dispziplin. \parencite[35ff.]{weber_music_2004} In der Folge waren musizierende Kinder auch ein probates Mittel, um in den Salons der Aristokratie einen guten Eindruck zu machen. Die Zusammenkünfte in den Salons waren in mehrerlei Hinsicht gesellscahftlich bedeutsam. Für junge, musizierende Mädchen war es mitunter ein guter Ort um Aufmerksamkeit auf sich zu lenken, die eventuell in einer Ehe mit einer bedeutenden Familie enden könnte. Auch die Zurschaustellung des eigenen Lebenstils, zum Beispiel durch die Wahl der Kleidung war ein wichtiger Bestandteil der Salsons.\footnote{Auf die ---- des Lebenstils wird im Abschnitt \ref{lebensstil} eingegangen werden.} Nicht zuletzt war der Erweb musikalischer Fähigkeiten, um sich in den Salons der angesehenen Familien bewegen zu können, teilweise auch ein Weg zu einer guten Anstellung: \blockquote[{\cite[37]{weber_music_2004}}]{A correspondent to a Leipzig music magazine reported that in Vieanna young men took up chamber music in order to gain access to the salons of thighly-placed families through whom they migth get good jobs.}

Auch war die Rolle der Frauen in diesen Kreisen insgesamt eine sehr bedeutsame. Frauen waren nicht nur von klein auf musikalisch tätig, was sie auch oft nach der Hochzeit fortsetzten. Sie waren auch für das gesellschaftliche Familienleben im Allgemeinen zuständig: Frauen organisierten die Zusammenkünfte in den Salons, stellten die Gäste vor, und führten die Konverstation. Auch die Tatsache, dass der Gesang und das Klavierspiel, zwei -- zumindest zur damaligen Zeit -- Domänen der Frauen, die beliebtesten musikalischen Formen in den Salons waren, spricht für die Bedeutung, die Frauen im musikalischen Leben der Familien spielten. \parencite[41]{weber_music_2004}

Bei der Betrachtung der Musikszene in den Salons könnte der Gedanke aufkommen, Musik sei nur als Mittel zum Zweck, als Möglichkeit, seine gesellschaftliche Position zu verbessern zu verstehen gewesen. Das trifft bei den öffentlichen Konzerte aus einem einleuchtenden Grund nicht zu: während man in den privaten Kreisen der Salons noch leicht einen Überblick über die anwesenden Personen behalten konnte, war das aufgrund der größer werdenen Publika bei den öffentlichen Konzerten, und der allgemeinen Zunahme der öffentlichen Konzerte nicht mehr möglich. \parencite[37]{weber_music_2004} Dies führt nun zur Frage, worauf diese Zunahme zurückzuführen ist. 

Wie in Kaptiel \ref{18jh} dargestellt, bestanden die frühen Formen des Konzerts aus einem Potpourri von verschiedenen Stücken, Arien, Tänzen und virtuosen Stücken. Das Liebhaberkonzert hatte anfangs eine ähnliche Zusammensetzung. Ausgehend vom Virtuosenkult um Instrumentalisten wie Franz Liszt oder Niccolò Paganini veranstalteten einzelne MusikerInnen Konzerte, in denen sie sich präsentieren konnten. Diese Konzerte waren mit der den musikalischen Aktivitäten in den privaten Kreisen eng verzahnt: einerseits traten die Solisten in den Salons auf, um für Unterhaltung zu sorgen, sowie teilweise auch um mit Familien in Kontakt zu kommen, deren Kinder sie dann unterrichten konnten. Auf der anderen Seite wurde dann von den Gesellschaften, in die sie eingeladen worden waren, erwartet, dass diese das regelmäßig (zum Beispiel jährlich) stattfindende öffentliche Konzert des Virtuosen zu besuchen. Aus der Nennung von Liszt oder Paganini wird klar, was das Besondere an den Konzerten war: es ging zentral um die Präsentation eines oder mehrerer Virtuosen, die mit ihren ungeahnten Fähigkeiten auf dem Instrument die Menschen begeisterten. \parencite[24]{weber_music_2004}

Trotzdem, und auch gerade weil die Konzerte eine große Anzahl an Menschen anzuziehen vermochten, entstand dadurch noch kein vereintes, elitäres Publikum, wie es später durch die Bildung von Symphonieorchestern verstärkt geschah. Durch die einzelnen MusikerInnen, die sich unternehmerisch betätigten, gab es eine solche Vielzahl an Konzerten, dass nicht alle von ihnen in der Lage waren, ein großes Publikum anzuziehen. In der Folge blieben die Konzertsäle oft teils leer, oder es wurden Karten zu geringen Preisen oder überhaupt gratis vergeben, um zumindest den Anschein eines großen Andranges zu erwecken. \parencite[49ff.]{weber_music_2004}

Ungeachtet dieser Einschränkungen bildeten die Liebhaberkonzerte aber eine zentrale Rolle in der Entwicklung des modernen Konzerts. Einerseits legten sie den Grundstein für Konzertformen wie das heute praktizierte Recital. Andererseits schufen sie durch die Vielzahl an Institutionen und Gewohnheiten die sie begünstigten, wie das musikalische Verlagswesen, die Musikpresse, und die allgemeine kommerzielle Ausrichtung der Konzerte, die Grundlage für die weiteren Entwicklungen des Konzertwesens.

\subsection{Die geschlossene Konzertvereinigung}
\label{konzertvereinigung}

\subsection{Die professionelle Konzertgesellschaft}
\label{konzertgesellschaft}

\section{Professionalisierung des Konzertbetriebs}
\label{professionalisierung}
\section{Der Lebensstil des Konzertpublikums}
\label{lebensstil}


\subsection{Kanonbildung in der klassischen Musik}

%%%%%%%%%%%%%%%%%%%%%%%%%%%%%%%%%%%%%%%%%%%%%%%%%%%%%%%%%%%%%%%%%%%%%%%%%%%%%%%%%%%%%%%%%%%%%%%%%%%%%%%%%%%%%%%%%%%%%%%%%%%%%%%%%%%%%%%%%%%%%%%%%%%%%%%%%%
%%%%%%%%%%%%%%%%%%%%%%%%%%%%%%%%%%%%%%%%%%%%%%%%%%%%%% Soziologischer Teil %%%%%%%%%%%%%%%%%%%%%%%%%%%%%%%%%%%%%%%%%%%%%%%%%%%%%%%%%%%%%%%%%%%%%%%%%%%%%%%
%%%%%%%%%%%%%%%%%%%%%%%%%%%%%%%%%%%%%%%%%%%%%%%%%%%%%%%%%%%%%%%%%%%%%%%%%%%%%%%%%%%%%%%%%%%%%%%%%%%%%%%%%%%%%%%%%%%%%%%%%%%%%%%%%%%%%%%%%%%%%%%%%%%%%%%%%%

\part{Soziologischer Teil}
\chapter{Bourdieu}

\chapter{Sozialstruktur}

\chapter{Empirische Erhebungen zum klassischen Konzertpublikum}

\part{Diskussion}
\chapter{Resümee}




\printbibliography
\end{document}

