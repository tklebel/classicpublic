\documentclass[a4paper, german, oneside]{scrbook}

\usepackage[german]{babel} %ngerman

\usepackage[utf8]{inputenc} 

\usepackage[T1]{fontenc}  
\usepackage{lmodern}

\usepackage{graphicx}

\usepackage{datetime}
\newdate{date}{15}{07}{2014}
\date{\displaydate{date}}

% Fußnote für Tabellen
\usepackage{tablefootnote}
\usepackage{amsmath}

% Auflistung in Text mit \begin{inparaenum}[(i)] und dann \item
\usepackage{paralist}
% Bessere Formatierung für Tabellen
\usepackage{booktabs}


\usepackage[babel,german=quotes]{csquotes}  % Deutsche Anführungszeichen mit \enquote{}

\usepackage[nottoc]{tocbibind}
\usepackage[style=authoryear, backend=biber, firstinits=false]{biblatex} % firstinits kürzt die Vornamen ab
\addbibresource{lit_ba_arbeit.bib}

\DefineBibliographyStrings{english}{andothers={et\ al\adddot}} % "u.a." zu "et al." 
% \DefineBibliographyStrings{english}{bibliography = {References}}
\DefineBibliographyStrings{german}{bibliography = {Literaturverzeichnis}}

% Internetquellen und sonstige Quellen separat
\defbibheading{LV}{\section*{Literaturverzeichnis}}
\defbibheading{IQ}{\section*{Internetquellen}}
 
\defbibfilter{LV}{\not\keyword{Internet}}
\defbibfilter{IQ}{\keyword{Internet}}
 


%richtige Reihenfolge bei mehreren Autoren
\DeclareNameAlias{sortname}{last-first}

%keine Klammern in Biblio
\renewbibmacro*{date+extrayear}{%
  \iffieldundef{year}
    {}
    {\printtext{\printdateextra}}}

\renewcommand*{\mkbibnamefirst}[1]{#1\addcomma} % #1 

% Commands für Textcite und Parencite mit Dropdownmenü
\newcommand{\citet}[1]{\textcite{#1}} 
\newcommand{\citep}[1]{\parencite{#1}}

% fix weird problem with unicode
\DeclareUnicodeCharacter{00A0}{ }




% % Blockzitat mit Anführungszeichen
% \renewcommand*{\mkblockquote}[4]{\enquote{#1}#2#4#3}

\begin{document}
	\begin{titlepage}
	
		\begin{center}


		% Oberer Teil der Titelseite:
		% \includegraphics[width=0.15\textwidth]{./logo}\\[1cm]    

		\textsc{\LARGE Universität für Musik und Darstellende Kunst Graz}\\[1.5cm]

		% Titel der LV
		% \textsc{\Large Multivariate Datenanalyse (SS 2014)}\\[0.5cm]


		% Title
		\newcommand{\HRule}{\rule{\linewidth}{0.5mm}}
		\HRule \\[0.4cm]
		{ \huge \bfseries Irgendwas mit Publikum}\\[0.4cm]

		\HRule \\[2.5cm] % Bei Untertitel auf 0.5 zurücksetzen


		% \Large Untertitel	\\[2cm]
		% Evtl Wortumfang der Arbeit hier einfügen

		\Large \emph{Autor}:\\
		Thomas \textsc{Klebel}\\[0.1cm]
		\large 1073073\\[1cm]

		\Large \emph{Betreuerin:}\\
		Christa \textsc{Brüstle}\\[1cm]

		% \includegraphics[scale=0.5]{uni-graz_color.jpg}\\[1cm]    


		\vfill

		% Unterer Teil der Seite
		\Large{\today}\\[1cm]
		\normalsize{Zeichensatz mit \LaTeX}
		

		\end{center}

	\end{titlepage}

	
%%%%%%%%%%%%%%%%%%%%%%%%
% zähle die Titelseite nicht für die Seitenzahl
\clearpage
\setcounter{page}{1}
%%%%%%%%%%%%%%%%%%%%%%

\tableofcontents

\chapter*{Einleitung}
\addcontentsline{toc}{chapter}{Einleitung}

Im ersten Teil der Arbeit soll die Entwicklung des klassischen Konzerts, insbesondere im 19. Jahrhundert nachvollzogen werden. 

Im zweiten Teil der Arbeit soll der/die LeserIn einerseits in die Theoriewelt Bourdieus eingeführt werden. Andererseits werde ich verschiedene Modelle zur Sozialstrukturananlyse darstellen, sowie aktuelle empirische Untersuchungen zum Konzertpublikum rezipieren. Die empirischen Befunde, sowie die in Teil 1 dargelegte Entwicklung des klassischen Konzerts, sollen dann mittels der vorgestellten Theorien untersucht werden. 

Die Analyse wird sich an folgenden Fragestellungen orientieren:

\begin{itemize}
	\item Welche Rolle spielte das Konzert im Leben der Bevölkerung des 19. Jahrhunderts? %viel zu allgemein
	\item Wie stellt sich das Verhältnis zwischen Konzertbesuch und sozialer Lage im 19. Jahrhundert dar?
	\item Gibt es verbindende Merkmale, die den KonzertbesucherInnen des 19. Jahrhunderts gemein sind?
\end{itemize}

Die Erkenntnisse aus der Untersuchung des Konzertpublikums des 19. Jahrhunderts sollen den empirischen Befunden über das heutige Konzertpublikum gegenüber gestellt werden:

\begin{itemize}
	\item Aus welchen Personen(gruppen) setzt sich das heutige Konzertpublikum zusammen?
	\item Welche Gemeinsamkeiten und Unterschiede lassen sich im Vergleich mit dem Konzertpublikum des 19. Jahrhunderts finden?
\end{itemize}

%%%%%%%%%%%%%%%%%%%%%%%%%%%%%%%%%%%%%%%%%%%%%%%%%%%%%%%%%%%%%%%%%%%%%%%%%%%%%%%%%%%%%%%%%%%%%%%%%%%%%%%%%%%%%%%%%%%%%%%%%%%%%%%%%%%%%%%%%%%%%%%%%%%%%%%%%%
%%%%%%%%%%%%%%%%%%%%%%%%%%%%%%%%%%%%%%%%%%%%%%%%%%%%%%%%%%%%%%%%% Historischer Teil %%%%%%%%%%%%%%%%%%%%%%%%%%%%%%%%%%%%%%%%%%%%%%%%%%%%%%%%%%%%%%%%%%%%%%
%%%%%%%%%%%%%%%%%%%%%%%%%%%%%%%%%%%%%%%%%%%%%%%%%%%%%%%%%%%%%%%%%%%%%%%%%%%%%%%%%%%%%%%%%%%%%%%%%%%%%%%%%%%%%%%%%%%%%%%%%%%%%%%%%%%%%%%%%%%%%%%%%%%%%%%%%%


\part{Historischer Teil}
% begingropu und endgroup umfassen einen Bereich ohne Pagebreak

\begingroup
\renewcommand{\cleardoublepage}{}
\renewcommand{\clearpage}{}

\chapter{Einleitung}
Um die Zusammensetzung des heutigen KOnzertpublikums zu untersuchen erscheint es sinnvoll, zuerst einen Blick auf die Genese des Konzerts zu werfen: Wann ist das Konzert, so wie wir es heute kennen, entstanden? Aus welchen Formen ist es entstanden, welche Vorläufer gab es? Auf Basis der Antworten auf diese Fragen lassen sich dann die heutigen Entwicklungen diskutieren.

Um einen guten Überblick über die Genese des Konzerts zu bieten, werden im ersten Abschnitt die Anfänge der Entwicklung zum Konzert betrachtet werden. Im zweiten Abschnitt wird es um das Musikleben im 19. JAhrhundert gehen. Neben einem kurzen historischen Überblick über die wichtigsten Geschehnisse dieses Jahrhunderts sollen, neben den Unterschieden zwischen dem Opernpublikum und dem Konzertpublikum, hautpsächlich die für die Entstehung des \enquote{bürgerlichen} Konzerts bedeutenden Formen dargestellt werden: 
\begin{inparaenum}[(1)]
	\item das Liebhaberkonzert, 
	\item die geschlossene Konzertvereinigung, sowie 
	\item die professionelle Konzertgesellschaft.
\end{inparaenum}



% \chapter{Das Musikleben im 18. Jahrhundert}
% \label{18jh}
% Mix aus verschiedenen Stücke, arien etc (Weber: 3)

% Aus heutiger PErspektive erstaunlich erscheint die Tatsache, dass das Liebhaberkonzert im Grunde ein Potpourri aus den verschiedensten WErken war. Typischerweise eröffnete es mit einem Satz aus einer Sinfonie, auf den Lieder, Arien, und virtuose Instrumentalstücke folgten. (Literatur!!!)


\endgroup

\chapter{Das Musikleben im 19. Jahrhundert}
\label{19jh}

\section{Historischer Überblick: Das 19. Jahrhundert}
\label{histUberblick}
Bei der Untersuchung der Musik und des Publikums lassen sich die historischen Ereignisse der Zeit nicht ignorieren, sie bilden vielmehr den Rahmen für alle Entwicklungen, die im weiteren Verlauf behandelt werden sollen. Insofern erscheint es angeraten, an den Anfang der Untersuchungen über das Musikleben im 19. JAhrhundert einen kleinen historischen ÜBerblick zu geben, der sich, ob des Rahmens der Arbeit, beispielhaft auf die wichtigsten Ereignisse konzentrieren wird.

Kalendarisch reichte das 19. Jahrhundert von 1801-1900. Trotzdem scheint es angebracht, den Rahmen etwas zu erweitern. Durch die Einbeziehung der französischen Revolution lässt sich ein besseres Bild der Entwicklungen darstellen.

Ausgehend von der französischen Revolution 1789 entwickelte sich Europa weg von monarchistischen Reichen, hin zu demokratischen Nationalstaaten. Nachdem auf die französische Revolution eine Zeit der Restauration folgte, gewann das Bürgertum mit den Revolutionen 1830 und 1848 deutlich an Einfluss. \parencite[vgl.][253ff.]{demandt_kleine_2003}

Ein Katalysator für den Aufstieg des Bürgertums war sicherlich die industrielle Revolution, und die daraus folgende Urbanisierung. Mit der Erfindung der Dampfmaschine im 18. Jahrhundert war der Grundstein für eine große Umwälzung der Wirtschaft, und in der Folge auch des Lebens der Menschen gelegt. Durch den Aufstieg der dampfgetriebenen Eisenbahn könnten Güter und Personen viel großräumiger bewegt werden. Insgesamt ging der Anteil der Beschäftigten im primären Sektor, der Agrar- und Landwirtschaft durch die technischen Fortschritte stark zurück, es entwickelte sich die \emph{Industriegesellschaft}. Sie war geprägt durch kapitalistische Warenproduktion. Die vom Land in die Städte ziehende Bevölkerung arbeitete unter schlechten Bedinungen in Fabriken, und trug, um mit Marx zu sprechen, zu einer Akkumulierung des Kapitals in den Händen der Unternehmer bei. (\cite{marx_kapital:_1989}; \cite[368]{hillmann_worterbuch_2007}) Der Aufstieg und die dadurch bedingte Eigenständigkeit des Bürgertums waren auch für die Entwicklungen im Musikleben bedeutend, wie wir in den nächsten Abschnitten sehen werden.


\section{Konzerttypen im 19. Jahrhundert}
\label{konzerttypen}
Bei der Analyse des Publikums lassen sich in der Literatur zwei Tendenzen erkennen. In den Werken von \citet{weber_music_2004} und \citet{bourdieu_feinen_2012} findet sich die klare Unterscheidung zwischen \emph{high culture} und \emph{popular culture} bei Weber, beziehungsweise \emph{legitimen}, \emph{mittlerem} und \emph{populärem Geschmack} bei Bourdieu. Bei neueren Untersuchungen, unter anderem bei \citet{gebesmair_grundzuge_2001} sowie \citet{muller_publikum_2014} wird diese Sicht teilweise kritisiert und relativiert. Ich möchte trotzdem mit einer Darstellung der Überlegungen bei Weber beginnen. Im späteren Verlauf der Arbeit sollen dann die Unzulänglichkeiten dieser Sichtweise diskutiert werden.

Liebhaberkonzert == benefit concert

\subsection{Das Liebhaberkonzert}
\label{liebhaber}
Obwohl der Terminus der \enquote{populären Musik} umstritten sein mag, so war doch das Liebhaberkonzert ein wichtiger Schritt auf dem Weg zur Form des Konzerts, so wie wir sie heute erleben. In der ersten HÄlfte des 19. Jahrhunderts gehörte es zu einer ----------, die maßgeblich an der Erweiterung des Publikums beteiligt war. Zu dieser ----- gehörten unter anderem eine institutionalisierte Form von Salons und Konzerten, die zusammen mit dem aufkeimendem Unternehmertum und der professionalisierung der MusikerInnen ein wachsendes Publikum aus der Aristokratie und der oberen Mittelschicht für sich gewinnen konnte.

Um die Rahmenbedinungen des Liebhaberkonzerts besser zu verstehen lohnt sich ein Blick in die Welt des Publikums dieser Konzerte.

Musik hatte für die aristokratischen Zirkel und die auf eine Verbesserung ihrer gesellschaftlichen Position bedachte obere Mittelschicht mehrere Funktionen. Ein erster Grund für die Tatsache, dass Musik eine immere wichtigere Rolle im Leben der Menschne einnahm war ihre Verankerung in der Familie. Zuerst einmal war eine musikalische ERziehung der Kinder ein guter Weg, sie Disziplin zu lehren: das ERlernen eines Musikinstruments fordert regelmäßige und konzentrierte Übung, folglich ein gewisses Maß an Dispziplin. \parencite[35ff.]{weber_music_2004} In der Folge waren musizierende Kinder auch ein probates Mittel, um in den Salons der Aristokratie einen guten Eindruck zu machen. Die Zusammenkünfte in den Salons waren in mehrerlei Hinsicht gesellscahftlich bedeutsam. Für junge, musizierende Mädchen war es mitunter ein guter Ort um Aufmerksamkeit auf sich zu lenken, die eventuell in einer Ehe mit einer bedeutenden Familie enden könnte. Auch die Zurschaustellung des eigenen Lebenstils, zum Beispiel durch die Wahl der Kleidung war ein wichtiger Bestandteil der Salsons.\footnote{Auf die ---- des Lebenstils wird im Abschnitt \ref{lebensstil} eingegangen werden.} Nicht zuletzt war der Erweb musikalischer Fähigkeiten, um sich in den Salons der angesehenen Familien bewegen zu können, teilweise auch ein Weg zu einer guten Anstellung: \blockquote[{\cite[37]{weber_music_2004}}]{A correspondent to a Leipzig music magazine reported that in Vieanna young men took up chamber music in order to gain access to the salons of thighly-placed families through whom they migth get good jobs.}

Auch war die Rolle der Frauen in diesen Kreisen insgesamt eine sehr bedeutsame. Frauen waren nicht nur von klein auf musikalisch tätig, was sie auch oft nach der Hochzeit fortsetzten. Sie waren auch für das gesellschaftliche Familienleben im Allgemeinen zuständig: Frauen organisierten die Zusammenkünfte in den Salons, stellten die Gäste vor, und führten die Konverstation. Auch die Tatsache, dass der Gesang und das Klavierspiel, zwei -- zumindest zur damaligen Zeit -- Domänen der Frauen, die beliebtesten musikalischen Formen in den Salons waren, spricht für die Bedeutung, die Frauen im musikalischen Leben der Familien spielten. \parencite[vgl.][41]{weber_music_2004}

Bei der Betrachtung der Musikszene in den Salons könnte der Gedanke aufkommen, Musik sei nur als Mittel zum Zweck, als Möglichkeit, seine gesellschaftliche Position zu verbessern zu verstehen gewesen. Das trifft bei den öffentlichen Konzerte aus einem einleuchtenden Grund nicht zu: während man in den privaten Kreisen der Salons noch leicht einen Überblick über die anwesenden Personen behalten konnte, war das aufgrund der größer werdenen Publika bei den öffentlichen Konzerten, und der allgemeinen Zunahme der öffentlichen Konzerte nicht mehr möglich. \parencite[vgl.][37]{weber_music_2004} Dies führt nun zur Frage, worauf diese Zunahme zurückzuführen ist. 

% nachfolgend einbinden, wie es im 18. Jh war, kein extra kapitel
Wie in Kaptiel \ref{18jh} dargestellt, bestanden die frühen Formen des Konzerts aus einem Potpourri von verschiedenen Stücken, Arien, Tänzen und virtuosen Stücken. Das Liebhaberkonzert hatte anfangs eine ähnliche Zusammensetzung. Ausgehend vom Virtuosenkult um Instrumentalisten wie Franz Liszt oder Niccolò Paganini veranstalteten einzelne MusikerInnen Konzerte, in denen sie sich präsentieren konnten. Diese Konzerte waren mit der den musikalischen Aktivitäten in den privaten Kreisen eng verzahnt: einerseits traten die Solisten in den Salons auf, um für Unterhaltung zu sorgen, sowie teilweise auch um mit Familien in Kontakt zu kommen, deren Kinder sie dann unterrichten konnten. Auf der anderen Seite wurde dann von den Gesellschaften, in die sie eingeladen worden waren, erwartet, dass diese das regelmäßig (zum Beispiel jährlich) stattfindende öffentliche Konzert des Virtuosen zu besuchen. Aus der Nennung von Liszt oder Paganini wird klar, was das Besondere an den Konzerten war: es ging zentral um die Präsentation eines oder mehrerer Virtuosen, die mit ihren ungeahnten Fähigkeiten auf dem Instrument die Menschen begeisterten. \parencite[vgl.][24]{weber_music_2004}

Trotzdem, und auch gerade weil die Konzerte eine große Anzahl an Menschen anzuziehen vermochten, entstand dadurch noch kein vereintes, elitäres Publikum, wie es später durch die Bildung von Symphonieorchestern verstärkt geschah. Durch die einzelnen MusikerInnen, die sich unternehmerisch betätigten, gab es eine solche Vielzahl an Konzerten, dass nicht alle von ihnen in der Lage waren, ein großes Publikum anzuziehen. In der Folge blieben die Konzertsäle oft teils leer, oder es wurden Karten zu geringen Preisen oder überhaupt gratis vergeben, um zumindest den Anschein eines großen Andranges zu erwecken. \parencite[vgl.][49ff.]{weber_music_2004}

Ungeachtet dieser Einschränkungen bildeten die Liebhaberkonzerte aber eine zentrale Rolle in der Entwicklung des modernen Konzerts. Einerseits legten sie den Grundstein für Konzertformen wie das heute praktizierte Recital. Andererseits schufen sie durch die Vielzahl an Institutionen und Gewohnheiten die sie begünstigten, wie das musikalische Verlagswesen, die Musikpresse, und die allgemeine kommerzielle Ausrichtung der Konzerte, die Grundlage für die weiteren Entwicklungen des Konzertwesens.

\subsection{Die geschlossene Konzertvereinigung}
\label{konzertvereinigung}

%fehlt: unterscheidung zwischen populärer und "klassischer" musik.  klar machen, dass bei weber liebhaberkonzerte in die erste Kategorie fallen.
Parallel zu den Entwicklungen im Rahmen der populären Musik (Weber) entstanden im 19. Jahrhundert auch so genannte Konzertvereinigungen. Am Beispiel der \enquote{Ancient Concerts} und der \enquote{Royal Philharmonic Society} in London lässt sich deren Wesen erläutern.

Das Publikum der Ancient Concerts bestandt anfangs fast ausschließlich aus Vertretern des hohen Adels und des Klerus. Nur auf Einladung durch die Direktoren, welche natürlich selbst dem hohen Adel angehörten, war es Personen gestattet, zu den Konzerten zu kommen. Entsprechend dem Geschmack der Elite wurden hautpsächlich Stück von Komponisten gespielt, die seit mindestens 20 Jahren verstorben waren. Faktisch bestand das Programm also aus Werken der Renaissance und des Barock. Im Lichte der zunehmenden Professionalisierung in anderen Bereichen des Musiklebens, sowie der steigenden Nachfrage nach \enquote{moderner} Musik, verlor die Konzertreihe allmählich an Reputation. (\cite[92ff.;]{muller_publikum_2014} \cite[73]{weber_music_2004})

Die Nachfrage des hohen Bürgertums nach aktuellerer Musik und größerer Unabhängigkeit von der Aristokratie wurde durch die Gründung der Royal Philharmonic Society, zumindest für eine kurze Periode, gestillt. Auch hier war die Mitgliedschaft nur durch die offizielle Aufnahme durch das Direktorium möglich. Im Gegensatz zu den Ancient Concerts machten Adlige hier aber den kleinere Teil aus. Der private Charakter der Veranstaltungen zeigte sich auch in der Tatsache, dass die Weitergabe der Eintrittskarten an Freunde strengstens untersagt war. \parencite[vgl.][93]{muller_publikum_2014}

Insgesamt war auch bei den geschlossenen Konzertvereinigungen die Tendenz der Professionalisierung zu beobachten. (vgl. Abschnitt \ref{professionalisierung}) Die Programme der Konzerte waren aber von denen der Liebhaberkonzerte grundverschieden. Während bei letzteren neue und virtuose Stücke für Begeisterung sorgten, waren es im Rahmen der Konzertvereinigungen und der noch zu besprechenden Konzertgesellschaften (vgl. Abschnitt \ref{konzertgesellschaft}) eher die \enquote{klassischen} Werke von Haydn, Mozart und Beethoven, die zur Aufführung gelangten. \parencite[vgl.][22f]{weber_music_2004}


\subsection{Die professionelle Konzertgesellschaft}
\label{konzertgesellschaft}
Aus den Konzertvereinigungen heraus entwickelten sich in der zweiten Hälfte des 19. Jahrhunderts Konzertgesellschaften, die immer professioneller organisiert wurden.\footnote{Der Prozess der PRofessionalisierung des Konzertbetriebes wird im nächsten Abschnitt besprochen.} Das Wesen der Konzertgesellschaften ähnelte schon sehr stark den heutigen Konzertgesellschaften (wie zum Beispiel der Grazer \emph{Musikverein für Steiermark} oder die \emph{Gesellschaft der Musikfreunde in Wien}). Die Zugehörigkeit zu der Gesellschaft war nicht mehr durch eine formelle Aufnahme beschränkt, es wurden vielmehr Eintrittskarten für eine Saison verkauft (Abonnement), im späteren Verlauf dann auch Einzelkarten. \parencite[vgl.][106]{heister_konzert:_1983}

Innerhalb dieser Konzertgesellschaften entstand auch eine zunehmende Kanonisierung des Repertoires: Die Sinfonie als gesamtes \emph{Werk} gewann an Bedeutung, sie wurde zum Kulminationspunkt der \emph{absoluten} Musik im Orchester. In der Folge ging auch die Praxis, nur einzelne Sätze aufzuführen deutlich zurück. \parencite[vgl.][231ff.]{muller_publikum_2014}



\section{Professionalisierung des Konzertbetriebs}
\label{professionalisierung}
Die Professionalisierung des Konzertbetriebs war ein bedeutender Schritt zur Entwicklung des modernen Konzertes. Für Symphonieorchester heute mehr oder weniger selbstverständlich Dinge wie ein/e DirigentIn oder mehrere Proben vor einer Aufführung waren im 19. Jahrhundert noch keine Selbstverständlichkeit. So war es beispielsweise üblich, dass die Leitung des Ensembles jeweils wechselte: jeweils unterschiedliche Musiker des Ensembles übernahmen für einzelne Konzerte die Leitung des Konzerts. \parencite[vgl.][68]{weber_music_2004} Alternativ war es auch möglich, dass entweder der/die KonzertmeisterIn oder auch der/die KomponistIn, zum Beispiel vom Klavier aus die Einsätze gab. Die Einführung einer fixen künstlerischen Leitung in Form eines/r DirigentIn führte verständlicherweise zu einer genaueren, ergo \enquote{besseren} Darbietung.\footnote{Die Fragen, was in diesem Zusammenhang als \enquote{gut} oder \enquote{besser} zu verstehen ist, ist beileibe keine eindeutige. Schon im 19. Jahrhundert war die Frage nach der \enquote{Werktreue} eine heiß diskutierte. Unstrittig bleibt aber, dass durch die Einführung eines/r DirigentIn im Orchester größere Ordnung herrschte, was der Qualität der Aufführungen, sofern man darunter eine akkurate Wiedergabe des Werkes versteht, sicherlich zuträglich war.}

Neben der \enquote{Erfindung} des/r DirigentIn (Müller) war auch die Einführung mehrer Proben ein wichtiger Schritt, um das Niveau der Aufführungen zu heben. Nicht zuletzt durch die steigende Anzahl von Proben bestanden die Ensembles zunehmend aus professionellen MusikerInnen, welche als Hauptbeschäftigung der Musik nachgingen. Diese Feststellung ist deshalb wichtig, da in den Anfängen des Konzertbetriebes die musizierenden Personen meist Amateure waren: sie gingen hauptberuflich einer anderen Tätigkeit nach, und \enquote{dilettierten} in ihrer Freizeit.

Am Begriff des \emph{Dilettanten} lässt sich dieser Wandel leicht zeigen. Der Duden listet für das Wort \enquote{Dilletant} zwei Bedeutungen:
\begin{inparaenum}[(a)]
	\item jemand, der sich mit einem bestimmten [künstlerischen, wissenschaftlichen] Gebiet nicht als Fachmann, sondern lediglich aus Liebhaberei beschäftigt, sowie
	\item (abwertend) jemand, der sein Fach nicht beherrscht.
\end{inparaenum}
\parencite{Dilettant}

Besonders die abwertende Verwendung des Wortes scheint heute die geläufige zu sein. Ganz im Gegensatz dazu steht seine Konnotation im 19. Jahrhundert: Der in seiner Freizeit dilletierende Bürger galt quasi als Ideal. Besonders in Wien waren die dilettierenden Musiker von großer Bedeutung und bremsten somit die Professionalisierung. \parencite[vgl.][88]{weber_music_2004}


% Sitzplatznummerierung

\section{Der Lebensstil des Konzertpublikums} %ev: die soziale Zusammensetzung des Publikums?
\label{lebensstil}
Wie schon aus den vorangegangenen Kapiteln ersichtlich rekrutierte sich das Publikum von klassischen Konzerten zur damaligen Zeit hauptsächlich aus den Reihen der Aristokratie sowie der oberen Mittelschicht. Innerhalb der verschiedenen Konzerttypen lässt sich aber eine weitere Differenzierung des Bürgertums nach der Berufsgruppe vornehmen.

Weber spricht von Konzerten der \enquote{popular music}, den Liebhaberkonzerten auf der einen Seite und denen der \enquote{classical music} auf der anderen Seite. In beiden Bereichen war das Bürgertum vertreten, nur eben durch unterschiedliche Berufsgruppen. Während in den Konzerten der \enquote{popular music} eher Unternehmer und ihre Familien zu finden waren, schienen die Konzerte der \enquote{classical music} eher eine Domäne der freien Berufe und der Beamten, in weiterer Konsequenz also der Männer zu sein. \parencite[vgl.][S. 63 und 66]{weber_music_2004}

In Verbindung mit den Berufsgruppen stand auch ihr Lebensstil. Nach Weber interessierten sich die Unternehmersfamilien besonders stark für die Virtuosen der \enquote{popular music}, da diese ihrem Lebensstil entsprachen: \blockquote[{\cite[39]{weber_music_2004}}]{Most business families accordingly had a self-confident, ostentatious lifestyle and valued cultural pursuits of the same order. They looked to the famous virtuosi for an exaggerated picture of the success and glamour they saw in themselves.}

Die Tatsache, dass Vertreter der freien Berufe sowie der Beamtenschaft bei den Konzerten der \enquote{classical music} stark vertreten waren, liegt in der Ausbildung dieser Berufsgruppen begründet: für Berufe wie den des Arztes oder des Rechtsanwaltes war, genauso wie für die Aufnahme in den Staatsdienst, eine universitäre Ausbildung eine Voraussetzung. In der Ausübung und Wertschätzung der schon damals \enquote{klassischen} Musik konnten sie ihren intellektuellen Habitus ausspielen und ihren Status verbessern. \parencite[vgl.][67]{weber_music_2004}




% \section{Kanonbildung in der klassischen Musik}

%%%%%%%%%%%%%%%%%%%%%%%%%%%%%%%%%%%%%%%%%%%%%%%%%%%%%%%%%%%%%%%%%%%%%%%%%%%%%%%%%%%%%%%%%%%%%%%%%%%%%%%%%%%%%%%%%%%%%%%%%%%%%%%%%%%%%%%%%%%%%%%%%%%%%%%%%%
%%%%%%%%%%%%%%%%%%%%%%%%%%%%%%%%%%%%%%%%%%%%%%%%%%%%%% Soziologischer Teil %%%%%%%%%%%%%%%%%%%%%%%%%%%%%%%%%%%%%%%%%%%%%%%%%%%%%%%%%%%%%%%%%%%%%%%%%%%%%%%
%%%%%%%%%%%%%%%%%%%%%%%%%%%%%%%%%%%%%%%%%%%%%%%%%%%%%%%%%%%%%%%%%%%%%%%%%%%%%%%%%%%%%%%%%%%%%%%%%%%%%%%%%%%%%%%%%%%%%%%%%%%%%%%%%%%%%%%%%%%%%%%%%%%%%%%%%%

\part{Soziologischer Teil}
\chapter{Distinktion und Habitus - Einführung in die Begriffswelt Bourdieus}
Bourdieus Theorien und Werke wurden und werden in der Soziologie breit rezipiert. Insbesondere in der Kultursoziologie haben seine Theorien noch immer einen großen Einfluss, so zum Beispiel auf die Theorie der Erlebnisgesellschaft nach \cite{schulze_erlebnisgesellschaft:_1993}. Auch die vorliegende Arbeit wird sich auf einige seiner Theorien und Begriffe beziehen. \parencite[vgl.][555]{joas_sozialtheorie:_2004}

\section{Kapitalsorten}
Aus der Gedankenwelt Marx' oder auch der Finanzwelt ist das \emph{Kapital} ein geläufiger Begriff. Es bezeichnet nach Marx Geld- und Sachwerte, die zur Produktion benötigt und verwendet werden können, beziehungsweise eher allgemein eine Geldmenge, über die man verfügen, und durch die man einen Gewinn erwirtschaften kann. \parencite{Kapital} Im Gegensatz zum Kapitalbegriff bei Marx versteht Bourdieu unter \enquote{Kapital} akkumulierte Arbeit. \parencite[vgl.][86]{luthje_medium_2008} Darüber hinaus hat er den eindimensionalen Bedeutungsrahmen des Begriffs um mehrere \emph{Kapitalsorten} erweitert. 


\subsection{Ökonomisches Kapital}
Das ökonomische Kapital ist am ehesten mit dem marxistischen Kapitalbegriff zu vergleichen. Es \blockquote[{\cite[86]{luthje_medium_2008}}]{ist materiell und dadurch gekennzeichnet, dass es sich problemlos, unmittelbar und direkt in Geld umwandeln lässt und sich damit besonders gut für die Institutionalisierung in der Form des Eigentumsrechts eignet.}

\subsection{Das kulturelle Kapital}
Bourdieu unterscheidet hierbei zwischen \emph{objektiviertem}, \emph{institutionalisiertem} und \emph{inkorporiertem} kulturellem Kapital. Unter \emph{objektiviertem Kapital} versteht er Kulturgüter, wie Musikinstrumente, Gemälde, Bücher und ähnliches. Das \emph{institutionalisierte} kulturelle Kapital besteht aus Titeln, die im Rahmen des Bildungsweges erworben wurden. Darunter fallen zum Beispiel Schulabschlüsse, Universitätsabschlüsse und sonstige institutionalisierte Forme von Bildungskapital. Schließlich beschreibt er aber auch noch \emph{inkorporiertes} kulturelles Kapital. Dieses bezeichnet die verinnerlichten Werte, Haltungen, die man durch die primäre Sozialisation und die weitere Schulbildung erworben hat. (vgl. \cite[539f.]{joas_sozialtheorie:_2004} sowie \cite[86f.]{luthje_medium_2008}) Insofern verweist es stark auf des Konzept des Habitus', auf welches im Abschnitt \ref{habitus} eingegangen wird.

\subsection{Soziales Kapital}
Unter sozialem Kapital sind die sozialen Beziehungen zu anderen Menschen gemeint, die man im Fall des Falles für sich nutzbar machen kann. Darunter fallen zum Beispiel \blockquote[{\cite[540]{joas_sozialtheorie:_2004}}]{die Zugehörigkeit zu einer bestimmten Gruppe, die Herkunft aus einer (angesehenen) Familie, [der] Besuch einer bestimmten Eliteschule} und ähnliches. Aus diesen Netzwerke lässt sich eventuell insofern \enquote{Kapital schlagen}, als man durch sie Zugang zu bestimmten Menschen, Jobs oder ähnlichem erhält.

\subsection{Symbolisches Kapital}
Schließlich spricht Bourdieu noch vom symbolischen Kapital, welches allgemein mit Prestige, gutem Ruf und Annerkennung gleichzusetzen ist, und insofern eine Art Oberbegriff für die anderen drei Kapitalsorten bildet, der selbige subsumiert. (vgl. \cite[540]{joas_sozialtheorie:_2004} sowie \cite[87]{luthje_medium_2008})

\subsection{Zusammenwirken der Kapitalsorten}
Die verschiedenen Kapitalsorten treten in den meisten Fällen nicht unabhänging voneinander auf. Jedes Individuum hat unterschiedlich große Mengen der verschiedenen Kapitalien zur Verfügung. So haben zum Beispiel, in einer idealtypischen Analyse, HochschullehrerInnen und und Industrie-, bzw. HandelsuntermehmerInnen ein ähnlich großes Kapitalvolumen, welches sich aber unterschiedlich zusammensetzt. Nach Bourdieu haben HochschullehrerInnen ein hohes ökonomisches Kapital, dafür ein geringeres ökonomisches Kapital, im Gegensatz zu VertreterInnen der Industrie, bei denen es umgekehrt ist. \parencite[vgl.][541]{joas_sozialtheorie:_2004}

$-->$ Grafik von Joas S. 541 einfügen

Plastischer darstellen lässt sich dieser Zusammenhang an einem Beispiel. Obwohl Studierende während des Studiums meist ein geringes ökonomisches Kapital zur Verfügung haben, erwerben sie eben durch ihr Studium eine Menge an anderem Kapital: Im Laufe ihres Studiums bauen sie Beziehungen und Netzwerke auf, die Menge an sozialem Kapital steigt also. Zugleich erwerben sie eine große Menge an Wissen, einerseits Faktenwissen, andererseits auch Wissen um die Regeln in einem gewissen Feld, sie bauen also ihr kulturelles Kapital aus. Nicht zuletzt durch den Erwerb eines (Bildungs-)Titels haben sie am Ende ihres Studiums eine große Menge an Kapital angesammelt, welches sie mit dem Eintritt in die Berufswelt in ökonomisches Kapital umwandeln können.

Diese Überlegungen führt Bourdieu im Konzept des Habitus zusammen, welches wohl einen seiner meist zitierten Begriffe darstellt.

% Was interessiert? Habitus und Geschmack

\subsection{Habitus}
\label{habitus}
Bourdieu geht davon aus, dass allen Menschen ein spezifischer Habitus innewohnt, der in Zusammenhang mit ihrer sozialen Lage steht. Unter Habitus versteht er hier Denk-, Wahrnehmungs-, und Verhaltensschemata, die durch die Sozialisation erlernt wurden, und vermittels derer man die Welt begreift nach ihnen handelt. Joas vermerkt dazu: \blockquote[{\cite[533]{joas_sozialtheorie:_2004}}]{Unsere Körperbewegungen, unser Geschmack, unsere banalsten Deutungen der Welt werden schon frühzeitig geformt und bestimmen dann in entscheidenem Ausmaß unsere weiteren Handlungsmöglichkeiten.} Keinesfalls ist damit eine völlige Determiniertheit durch die Zugehörigkeit zu einer Klasse gemeint. Bourdieu vertritt vielmehr die Ansicht, dass die durch die Sozialisation verinnerlichten, und damit auch inkorporierten, Handlungs-, und Verhaltensweisen relativ stabil bleiben, und somit auch eindeutig zuordenbar sind.

In seinem Werk \emph{Die feinen Unterschiede. Kritik der gesellschaftlichen Urteilskraft} untersucht er ausführlich den Geschmack und Lebensstil der französischen Bevölkerung. 

% ausführlicher darstellen: legitimen geschmack, diskurs um hohe kunst, inwiefern hängt das mit dem lebensstil zusammen. leitet über zum nächsten abschnitt

\section{Oper und Konzert - unterschiedliche Distinktionspraktiken im Fokus}
\label{operUndKonzert}

\chapter{Sozialstruktur} %vielleicht auslassen?

\chapter{Empirische Erhebungen zum klassischen Konzertpublikum}

\part{Diskussion}

% % Begründung für die These, dass Konzert früher stärkere Repräsentationsfuktion hatte...
% % der folgende Teil passt besser zur soziologischen Analyse
% Konzert als Mittel zur Distinktion
% Aufzählung der Gäste (müller 91)
% Publizierung der Mitgliederlisten


\chapter{Resümee}
blbabla



\printbibliography[filter=LV]
\printbibliography[heading=IQ, filter=IQ]
\end{document}

